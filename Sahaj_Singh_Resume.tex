\documentclass[letterpaper,11pt]{article}
    \usepackage{times}
    \usepackage{latexsym}
    \usepackage[empty]{fullpage}
    \usepackage{titlesec}
    \usepackage{marvosym}
    \usepackage[usenames,dvipsnames]{color}
    \usepackage{verbatim}
    \usepackage{enumitem}
    \usepackage[hidelinks, colorlinks=false]{hyperref}
    \usepackage[utf8]{inputenc}
    \usepackage[T1]{fontenc}
    \usepackage{fancyhdr}
    \usepackage[english]{babel}
    \usepackage{tabularx}
    \input{glyphtounicode}

\pagestyle{fancy}
\fancyhf{}
\fancyfoot{}
\renewcommand{\headrulewidth}{0pt}
\renewcommand{\footrulewidth}{0pt}
\setlength{\footskip}{4.08003pt}

% Adjust margins
\addtolength{\oddsidemargin}{-0.5in}
\addtolength{\evensidemargin}{-0.5in}
\addtolength{\textwidth}{1in}
\addtolength{\topmargin}{-.5in}
\addtolength{\textheight}{1.0in}

\urlstyle{same}

\raggedbottom{}
\raggedright{}
\setlength{\tabcolsep}{0in}

% Ensure that generate pdf is machine readable/ATS parsable
\pdfgentounicode=1

%----Resume Definitions----%

% Sections formatting:
\titleformat{\section}{
    \vspace{-10pt}\scshape\raggedright\large
}{}{0em}{}[\color{black}\titlerule{}\vspace{-5pt}]

% Custom commands:
\newcommand{\lineunder} {
    \vspace*{-8pt} \\
    \hspace*{-18pt} \hrulefill\\
}

\newcommand{\contact} [5] {
    \begin{center}
        \textbf{\Huge #1} \\ \vspace{1pt}
        \small \href{https://#2}{#2} $|$ \href{mailto:#3}{#3} $|$ 
        \href{#4}{\underline{LinkedIn}} $|$
        \href{#5}{\underline{GitHub}}
    \end{center}
}

\newcommand{\Item} [1] {
    \item\small{{#1 \vspace{-2pt}}}
}

\newcommand{\employer} [5] {
    {\textbf{#3} \hfill \textbf{#4 --- #5}\\ \textbf{\emph{#1}} (#2)\\}
}

\newcommand{\school} [5] {
    {\textbf{#3} \hfill \textbf{#4 --- #5}\\ \textbf{\emph{#1}} (#2)\\}
}

\newcommand{\workItemListStart} [0] {
    \vspace{-1pt}
    \begin{itemize}[leftmargin=*,topsep=0pt,itemsep=-2pt]
}

\newcommand{\workItemListEnd} [0] {
    \end{itemize}
    \vspace{1pt}
}

\newcommand{\resumeItemListStart} [0] {
    \vspace{3pt}
    \begin{itemize}[leftmargin=*,topsep=0pt,itemsep=-2pt]
}

\newcommand{\resumeItemListEnd} [0] {
    \end{itemize}
    \vspace{3pt}
}
%----End of Resume Definitions----%

\begin{document}
    \vspace*{-30pt}

    %----Profile----%
    % chktex-file 8
    \contact{Sahaj Singh}{sahajs.com}{sahaj\_singh@sfu.ca}{https://www.linkedin.com/in/sahaj--singh/}{https://github.com/SatireSage}
    %----End of Profile----%

    %----Skills----%
    \section{Technical Skills}
    \begin{itemize}[leftmargin=0.15in, label={}]
    \small{\item{
        \textbf{Programming Languages:}{ C, C++, Python, MATLAB, Java, Assembly, HTML5/CSS3, Javascript} \\
        \textbf{Tools \& Technologies:}{ Git VC, Visual Studio Code, PyCharm, macOS, Linux, Windows, Android} \\
    }}
    \end{itemize}
    %----End of Skills----%

    %----Experience----%
    \section{Work Experience}
    \employer{MathWorks}{Burnaby, BC}{MATLAB --- SFU Student Ambassador}{Oct 2022}{Present}
    \workItemListStart{}
        \Item{Organizing and hosting numerous programming and simulation based events revolving around MATLAB and Simulink.}
        \Item{Helping in the process of creating meaningful relationships between MATLAB and professors/students at SFU.\@ Providing support for students with questions related to MATLAB and Simulink.}
    \workItemListEnd{}
    \employer{picoTera Electronics Inc.}{Richmond, BC}{Software/Firmware Developer}{Jan 2022}{April 2022}
    \workItemListStart{}
        \Item{Developed advanced firmware in C/C++ for PSoC6 and ARM Cortex-M4, M0 platforms and ported the project from PSoC creator to ModusToolbox 2.4 for better compatibility.}
        \Item{Constructed a unique audio dataset for a Recurrent Neural Network (RNN) model, significantly improving its noise reduction capabilities in denoising applications.}
        \Item{Implemented Static Gain replacing dynamic gain, optimizing post-processing audio quality, and boosting denoising performance.}
        \Item{Authored custom cmake scripts for CMSIS libraries creation, reducing memory usage in complex operations and enabled Bluetooth Low Energy (BLE) integration between PSoC6 and an Android app, facilitating real-time data transmission.}
    \workItemListEnd{}
    %----End of Experience----%

    %----Projects----%
    \section{Projects}
    \href{https://github.com/SatireSage/Multi-threaded-Memory-Allocator}{\underline{\textbf{Multi-threaded Memory Allocator:}}} {\sl C, Make\/} \hfill \textbf{Spring 2023}
    \resumeItemListStart{}
        \Item{Developed a multi-threaded memory allocator in C, supporting First Fit, Best Fit, and Worst Fit allocation algorithms.}
        \Item{Implemented features such as allocator initialization, allocation/deallocation interfaces, metadata management, compaction support, statistics reporting, multi-threading support, and uninitialization.}
        \Item{Designed test cases and provided usage instructions to ensure the proper functionality and efficiency.}
    \resumeItemListEnd{}
    \href{https://github.com/SatireSage/FPGA-UART-Protocol}{\underline{\textbf{FPGA UART Protocol Implementation:}}} {\sl VHDL, Modelsim, Altera DE2\/} \hfill \textbf{Spring 2023}
    \resumeItemListStart{}
        \Item{Developed a UART protocol for the Altera DE2 FPGA, featuring baud rate generation, data framing, error detection and correction, and handshaking subsystems.}
        \Item{Implemented in VHDL with comprehensive testbenches to verify the functionality of both the transmitter and receiver modules.}
        \Item{Enabled synchronous data transmission between UART devices and allowed for seamless operation via onboard switches and keys for data input, baud rate selection, and module reset.}
    \resumeItemListEnd{}
    \href{https://github.com/SatireSage/Dronee}{\underline{\textbf{Drone Controller and System:}}} {\sl Embedded C, Arduino (C++), Beaglebone\/} \hfill \textbf{Fall 2022}
    \resumeItemListStart{}
        \Item{Designed a wireless drone system using a BeagleBone Green and Arduino Nano 33 IOT drone.}
        \Item{Developed multiple control modes, integrated LCD display, and implemented an ultrasonic sensor for gesture-based height control.}
        \Item{Wrote a custom driver for efficient BLE communication between the controller and drone.}
        \Item{Incorporated a watchdog and systemd script for automated restarts to handle any unexpected system crashes.}
    \resumeItemListEnd{}
    \href{https://github.com/SatireSage/FASTrack}{\underline{\textbf{FASTrack - Reaction Time Game:}}} {\sl Assembly, Embedded C, Zedboard\/} \hfill \textbf{Summer 2022}
    \resumeItemListStart{}
        \Item{Developed a reaction time-based game called FASTrack for the Xilinx ZedBoard, utilizing the ARM7 assembly instruction set.}
        \Item{Implemented various game features such as multiple speed modes and user controls through switches. Utilized OLED display and LEDs for visual feedback.}
        \Item{Demonstrated key concepts including timer interrupts, masking, OLED display usage, and Finite State Machines (FSMs).}
    \resumeItemListEnd{}
    %----End of Projects----%

    %----Education----%
    \section{Education}
    \school{Simon Fraser University}{B.A.Sc. Computer Engineering --- Honours}{Burnaby, BC}{Sep 2020}{Sep 2025}
    %----End of Education----%

    %----Awards----%
    \section{Awards}
    \textbf{Innovation Award}/\textbf{ESSS Award: 2022} \hfill SFU --- Engineering Science Student Society\\
    Honors contributions to the student society and Recognized for outstanding creativity, originality, and impact via my projects driving advancements in technology.\\
    %----End of Awards----%

\end{document}